\documentclass[conference]{IEEEtran}
\hyphenation{op-tical net-works semi-conduc-tor}
\usepackage{amsthm,amssymb,amsmath}
\usepackage{algorithmic}
\usepackage{graphicx}
\usepackage{bm}
\usepackage{array}
\usepackage{cite}
\usepackage{url}
\usepackage{gensymb}
\usepackage[justification=centering]{caption}

\begin{document}

\title{An Open Source Covariate based Software Reliability Assessment Tool}

\author{\IEEEauthorblockN{Author1 Name$^1$, Author2 Name$^1$, and Lance Fiondella$^1$}
\IEEEauthorblockA{$^1$Electrical and Computer Engineering, University of Massachusetts Dartmouth, MA, USA\\
	%}
%\IEEEauthorblockA{$^2$Naval Air Systems Command, Patuxent River, MD, USA\\
Email: \{lfiondella\}@umassd.edu}
}
%\and
%\IEEEauthorblockN{Thierry Wandji}
%\IEEEauthorblockA{Naval Air Systems Command}
%	Email: }
%Telephone: }

\maketitle

\begin{abstract}%Need to be revised
This paper presents an open source covariate based software reliability tool to automatically apply methods from software reliability engineering to incorporate covariate data to enable inferences about the software testing process. The tool provides functionalities including visualization of failure data, application of software reliability growth models including various hazard rate funtions, inferences made possible by these models, and assessment of goodness of fit. The application and source code are available through the web. The open source nature of this tool will enable unprecedented levels of collaboration among researchers and practitioners from industry and government within a single shared platform.
\end{abstract}

\begin{IEEEkeywords}
Software reliability, software reliability growth model, covariates, Python programming language, GitHub
\end{IEEEkeywords}


\section{Software Reliability Engineering}
What is NHPP SRGM? Why Covariate NHPP SRGM?

Existing tools. Reason for developing this tool.

Contribution.


Section~\ref{sec:SRAT} provides a brief overview of the tool's user interface, which is divided into three primary activities, including selection, analysis, and filtering Data~\ref{sec:Tab1}, model fit and prediction ~\ref{sec:Tab2}, and model evaluation~\ref{sec:Tab3}. Section~\ref{sec:Concl} concludes the paper and identifies future work.

\section{Covariate based Software Reliability Assessment Tool (CSRAT)}\label{sec:SRAT}
This section presents detailed discussion of the functionalities implemented in the tool and how to use it.

\subsection{Tab1: }\label{sec:Tab1}
Description

Figure 

%\begin{figure}[!h]
%\centering
%\includegraphics[width=3.5in]{Figures/FigureName}
%\caption{Tab one options}
%\label{fig:xxx}
%\end{figure}


\subsection{Tab2: }\label{sec:Tab2}
Description

Figure 

\subsection{Tab3: }\label{sec:Tab3}
Description

Figure 

\section{Conclusion and Future Research}\label{sec:Concl}
This paper presents an open source...

\section*{Acknowledgment}
This material is based upon work supported by the National Science Foundation under Grant Number (\#1749635). Any opinions, findings, and conclusions or recommendations expressed in this material are those of the authors and do not necessarily reflect the views of the National Science Foundation.

\bibliographystyle{IEEEtran}
\bibliography{bibIEEE}


\end{document}


