\documentclass[conference]{IEEEtran}
\hyphenation{op-tical net-works semi-conduc-tor}
\usepackage{amsthm,amssymb,amsmath}
\usepackage{algorithmic}
\usepackage{graphicx}
\usepackage{bm}
\usepackage{array}
\usepackage{cite}
\usepackage{url}
\usepackage{gensymb}
\usepackage[justification=centering]{caption}

\begin{document}

\title{An Open Source Covariate based Software Reliability Assessment Tool \\
	 A Guide for Contributors}

\author{\IEEEauthorblockN{Author1 Name$^1$, Author2 Name$^1$, and Lance Fiondella$^1$}
\IEEEauthorblockA{$^1$Electrical and Computer Engineering, University of Massachusetts Dartmouth, MA, USA\\
	%}
%\IEEEauthorblockA{$^2$Naval Air Systems Command, Patuxent River, MD, USA\\
Email: \{lfiondella\}@umassd.edu}
}
%\and
%\IEEEauthorblockN{Thierry Wandji}
%\IEEEauthorblockA{Naval Air Systems Command}
%	Email: }
%Telephone: }

\maketitle

\begin{abstract}%Need to be revised
This paper presents the application architecture of xxx, a free and open source application to promote the
\end{abstract}

%\begin{IEEEkeywords}
%Software reliability, software reliability growth model, covariates, Python programming language, GitHub
%\end{IEEEkeywords}


\section{Introduction}\label{sec:Intro}
Do not worry about this for now

%Section~\ref{sec:SRAT} provides a brief overview of the tool's user interface, which is divided into three primary activities, including selection, analysis, and filtering Data~\ref{sec:Tab1}, model fit and prediction ~\ref{sec:Tab2}, and model evaluation~\ref{sec:Tab3}. Section~\ref{sec:Concl} concludes the paper and identifies future work.

\section{Software Architecture}\label{sec:Architecture}
This section presents...

The covariate tool graphical user interface is written in Python using PyQt5, which provides Python bindings for the Qt libraries. The tool is available to download from xxx.
% not sure the final state of the tool; if its just to be downloaded for github or if we can release an executable. 

Need to have PyQt5 libraries installed

The tool accepts failure data input in Excel (.xls, .xlsx) or comma separated value (.csv) formats. The first column of the table must contain failure times and the second column must contain the number of failures at that time. The third column must contain covariate data, with any additional columns containing data for additional covariates.

% type of analysis?
The provided data can be analyzing using any subset of the covariates, including zero. Analysis can be performed using different hazard functions, of which the tool includes three: Geometric, Negative Binomial (Order 2), and Discrete Weibull (Order 2).
% Goodness of fit measures (AIC, BIC, etc..)
After analysis is performed, several goodness of fit measures are calculated and displayed in a table. These include AIC, BIC, ... , and xxx.

[example data sets, user guide links?]



Functions of tool: fitting models to data, comparing goodness of fit, prediction

\subsection{xxx}\label{sec:Tab1}
%Description
Shows list of loaded models to choose

View imported data as graph or table

Select covariates to perform measurements on once data is loaded. Reads covariate names from data file. If no header, generic names (Metric1, Metric2, etc..) given to covariates. 

Figure

%\begin{figure}[!h]
%\centering
%\includegraphics[width=3.5in]{Figures/FigureName}
%\caption{Tab one options}
%\label{fig:xxx}
%\end{figure}


\section{Model Specifications}\label{sec:Specs}
%Description
Generic model class contained in model.py in the core directory. Most definitions and methods are in model.py. Specific models are stored in the models directory, where one each model has its own file. This file can have any title, although for ease of use it is best to title the file the name of the model. The specific model must contain: an import of the generic Model class (from core.model import Model), the class must inherit from Model, a string containing the name of the model assigned to variable name, the initialization method (always exactly the same), and a calcHazard function. The calcHazard function takes one argument (b), and returns a list of values of the function evaluated for each discrete time of the imported failure data. 

% import generic Model class

% inherit from Model

% name = "Model Name"

% __init__() method (THE SAME FOR EVERY MODEL)

% calcHazard() method
%		where the unique hazard function is implemented

Code

\section{Model Integration and Testing}\label{sec:Integration}
xxx

\section{Conclusion and Future Research}\label{sec:Concl}
This paper presents an open source...

\section*{Acknowledgment}
This material is based upon work supported by the National Science Foundation under Grant Number (\#1749635). Any opinions, findings, and conclusions or recommendations expressed in this material are those of the authors and do not necessarily reflect the views of the National Science Foundation.

\bibliographystyle{IEEEtran}
\bibliography{bibIEEE}


\end{document}